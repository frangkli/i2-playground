\documentclass[letterpaper,11pt]{article}

\usepackage[shortlabels]{enumitem}
\usepackage[margin=1in]{geometry}
\usepackage[most]{tcolorbox}

\begin{document}
\title{{\bf Unit 3: Basic Neuroanatomy Project} }
\author{Name: Frank Li}

\date{}
\maketitle

\section*{Short Answer}
\begin{enumerate}[a)]
\item Describe several advantages and disadvantages of biological computation with the brain compared to machine learning

\begin{tcolorbox}
Advantages. The brain is more flexible in learning. We all know that brains are able to learn a variety of things concurrently and without any prior preparation. On the other hand, machine learning requires intricate models to be designed for specific topics of learning. Plus, the brain is more energy efficient because it is able to learn complex things with human energy, while complex models in machine learning like GPT requires thousands of watts of energy for servers to compute and learn.

Disadvantages. The brain has less computational power compared to machine learning super computers. This is hard to quantify since the human brain is not well understood but the amount of data and computation that super computers can do is unmatched by that of the human brain. Therefore, if we are able to find a more efficient algorithm for machine learning, the learning speed and efficiency of computers should be significantly better than the human brain. Moreover, machine learning results can be transferred easily. The learned model can be simply copied and transferred to another computer to transfer the knowledge to another entity, while human will have to go through the teach-and-learn process.
\end{tcolorbox}

\item Speculate what aspects of the architecture of the brain may cause these advantages or disadvantages, and similarly comment on aspects of machine learning’s architecture

\begin{tcolorbox}
The amount of neurons and interconnection between those neurons make the biological computation of information speedy. I believe this also contributes to the adaptability of the brain (it is able to learn many features in parallel and find connection between learning results). To be direct, it looks as if the brain is one giant adaptable and flexible machine learning model that is fit for learning all knowledge, while current machine learning models often focus on one specific goal such as image classification. Therefore, architecturally, the brain's billions of neurons and the connections create a large fully-connected network that is far more powerful (in terms of learning) compared to all machine learning models. Another key physical feature of the brain is that it fits in our heads supported by our necks, which means it has to be small and portable. This limitation does not exist for large computer clusters that machine learning models are ran on, therefore, they are able to be more computationally powerful and have more data stored in one location, just because of the lack of size limitation.
\end{tcolorbox}

\item Brainstorm some marvelous schemes for integrating advantages from both ways of computing. Draw, write, scribble etc… When you are done, do a quick google for your best ideas to see if anyone has researched or tried them already!

\begin{tcolorbox}
I think neuromorphic computing is very promising where the computer and algorithm architecture is supposed to mimic the brain. I think that field can help us understand the brain more and how to optimize our brain and computers to be most efficient in learning. Another field which I think is promising and interesting is brain-computer interfaces. This is the field that I have a lot of interest in because it allows for I/O and interactive computation between the brain and computer. The forefront of this research that I am acquainted to would be neurallink, but image an application where the BCI technology is able to fully integrate and connect the brain with external computers to combine the advantages of the brain with advantages of the computers.
\end{tcolorbox}
    
\end{enumerate}
\end{document}