\documentclass[letterpaper,11pt]{article}

\usepackage[shortlabels]{enumitem}
\usepackage[margin=1in]{geometry}
\usepackage[most]{tcolorbox}

\begin{document}
\title{{\bf Module 2: Subfield Selection} }
\author{Name: Frank Li}

\date{}
\maketitle

\section{Intro}
Now that you have chosen a subfield, it's time to get to work! This worksheet is a guide for how to approach deep diving into one subfield as a group, from which you will create a research question and project. Make sure to spend time on it! The time you spend now reading and thoroughly understanding the field will pay dividends when you create a project -- you will spend exponentially less time following footsteps and more time creating new things. Therefore, you will be reading at least TWO papers in your selected subfield. I highly recommend reading more if you feel inspired!
\newline
\newline
There won't be many questions this week, but rather a focus on reading and understanding. A pro tip: a huge portion of your tuition is spent on resources that allow you to access the frontiers of knowledge. UW has subscriptions to most prominent journals like Nature and ScienceDirect for example (as well as the NYT and WSJ!) and the UW libraries are an absolute WEALTH of information. It's a gift (and your tuition) so let's use it!!
\section{Understanding the Field}


\begin{enumerate}
    \item First, glance at the presentation spec that you will use to make a presentation next week. You do not have to do this yet, but you should know what's coming up.
    \item 
    To begin, please find at least 3 researchers in this subfield, and list their contact information here (email, or maybe LinkedIn). One tip that you should know is that researchers LOVE to talk about their papers, and if you email them with well thought out questions, you will likely get a response. A second tip is that their emails are often listed at the top of papers they write, especially in arXiv.
\begin{tcolorbox}
  \begin{itemize}
    \item Mark Weber, https://www.linkedin.com/in/markrweber/
    \item Aldo Pareja, https://www.linkedin.com/in/aldo-pareja/
    \item Jure Leskovec, https://cs.stanford.edu/people/jure/
  \end{itemize}
\end{tcolorbox}

\item 
Make a list of at least ten specific papers in the subfield here. Next, \textbf{meet with your group}, and choose at least two papers each to read. You may have multiple members read the same paper or you may all read different papers. Finally, summarize \textbf{both} papers' findings or arguments, add key takeaways and vocabulary you learned, and add questions you still have here!
\begin{tcolorbox}
  \begin{itemize}
    \item Ayle, Morgane, et al. Training Differentially Private Graph Neural Networks with Random Walk Sampling. arXiv:2301.00738, arXiv, 2 Jan. 2023. arXiv.org, https://doi.org/10.48550/arXiv.2301.00738.
    \item Fang, Hongjian, et al. Brain-Inspired Graph Spiking Neural Networks for Commonsense Knowledge Representation and Reasoning. arXiv:2207.05561, arXiv, 11 July 2022. arXiv.org, https://doi.org/10.48550/arXiv.2207.05561.
    \item Ju, Haotian, et al. Generalization in Graph Neural Networks: Improved PAC-Bayesian Bounds on Graph Diffusion. arXiv:2302.04451, arXiv, 9 Feb. 2023. arXiv.org, https://doi.org/10.48550/arXiv.2302.04451.
    \item Kapanipathi, Pavan, et al. Infusing Knowledge into the Textual Entailment Task Using Graph Convolutional Networks. arXiv:1911.02060, arXiv, 21 Nov. 2019. arXiv.org, http://arxiv.org/abs/1911.02060.
    \item Paliotta, Daniele, et al. Graph Neural Networks Go Forward-Forward. arXiv:2302.05282, arXiv, 10 Feb. 2023. arXiv.org, https://doi.org/10.48550/arXiv.2302.05282.
    \item Thangamuthu, Abishek, et al. Unravelling the Performance of Physics-Informed Graph Neural Networks for Dynamical Systems. arXiv:2211.05520, arXiv, 2 Feb. 2023. arXiv.org, https://doi.org/10.48550/arXiv.2211.05520.
    \item Weber, Mark, et al. Anti-Money Laundering in Bitcoin: Experimenting with Graph Convolutional Networks for Financial Forensics. arXiv:1908.02591, arXiv, 31 July 2019. arXiv.org, http://arxiv.org/abs/1908.02591.
    \item Werner, Luisa, et al. Knowledge Enhanced Graph Neural Networks. arXiv:2303.15487, arXiv, 27 Mar. 2023. arXiv.org, https://doi.org/10.48550/arXiv.2303.15487.
    \item Zhang, Ge, et al. Graph-Level Neural Networks: Current Progress and Future Directions. arXiv:2205.15555, arXiv, 31 May 2022. arXiv.org, https://doi.org/10.48550/arXiv.2205.15555.
    \item Bessadok, Alaa, et al. Graph Neural Networks in Network Neuroscience. arXiv:2106.03535, arXiv, 28 Sept. 2022. arXiv.org, http://arxiv.org/abs/2106.03535.
  \end{itemize}
\end{tcolorbox}

\end{enumerate}



\end{document}