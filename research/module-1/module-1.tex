\documentclass[letterpaper,11pt]{article}

\usepackage[shortlabels]{enumitem}
\usepackage[margin=1in]{geometry}
\usepackage[most]{tcolorbox}

\begin{document}
\title{{\bf Module 1: Subfield Exploration} }
\author{Name: Frank Li}

\date{}
\maketitle

\section{Intro}
This worksheet will help you explore various topics you generated last week in more depth and guide you towards choosing a research topic that is meaningful to you.
\subsection{Why Research Anything?}
Before we begin, I want you to spend a little bit thinking about why you are in this particular place spending your time on this particular activity. Write a paragraph or more about what you want to see as a result of taking on this intellectual challenge, about what is important to you, and how being here and participating in this relates. Maybe you don't know all the answers yet. As long as you have questions that's ok. Honestly, I never stop asking these questions. I never arrive at one answer, but I feel like I continually respond to them as I keep living. Have fun with it!
\begin{tcolorbox}
    I think my interest in research is primarily fueled by my curiosity in the world around me and the
    dislike of have spoonfed answers. This is not just about AI and neuroscience but also academics in general,
    I have a lot of competing interests and ideally I would like to explore them all in some way, but obviously
    time is limited. For example, I am also interested in philosophy, linguistics, and physics, but I simply do
    not have enough time and energy to pursue them all. Therefore, I had to choose some areas to focus on and
    since I chose AI as one of the key areas of focus, I want to perfect it as much as possible.
    I guess part of this is definitely coming from the ego of wanting to be knowledgeable at something and not
    rely on others for research breakthrough, but it is also completely true that I want to contribute to the
    world. And if there is one thing that I learned throughout my life, it is that contribution and impact is
    very difficult to achieve on a large scale, and I believe research is one of the best ways where I can
    spread my impact and influence. Last thing would be the fact that there is nothing more boring about fake
    exploration and discovery - where I ``research'' something but in reality the answer is just one google
    search away. I want to be a producer of knowledge instead of a sheer consumer, and I think many people in
    research share this sentiment about where they belong in this world and hence I like research and the
    research community.
\end{tcolorbox}
\pagebreak
\section{Exploration}
Using the list of interests from Worksheet 0, with your group assign one topic per 1 or 2 people. Then fill out the following questions about this subfield!

\begin{enumerate}[a)]
\item
    Provide a brief introduction to your chosen topic in ML/Neuro/Cog Sci. research.
    Define any key terms or concepts that are important for understanding your topic.

\begin{tcolorbox}
    The topic I chose is \textbf{graph neural network} (GNNs). GNNs are a class of neural networks that
    operate on graphs. They typically work by taking in a graph data structure input (which can be anything,
    even images can be represented as graphs) and output some sort of observation or prediction regarding
    the relationship between the nodes in the provided graph. I like to think of GNNs as a generalization
    of CNNs, where CNNs operate on images and GNNs operate on graphs. The reason why I chose GNNs is
    because I think it is a very interesting topic as though graphs are used a lot in software engineering,
    they are not used as much in ML.
\end{tcolorbox}

\item     List some real-world applications of your chosen topic.
    Explain how your chosen topic can be applied in each.

\begin{tcolorbox}
    Social/relational network modelling for social media or knowledge graphs.\\
    Word embedding and relationship mapping for natural language processing.\\
    Optimization of traditional graph algorithms such as path finding.\\
    There may also be some applications in CV and RL using graph representations of images and states.
    Anything representable as a graph could be used as input for GNNs, so the applications of GNNs
    are limitless.
\end{tcolorbox}

\item    List some potential future directions for research in your chosen topic.
    Explain why these future directions are important and what they could contribute to the field of machine learning.

\begin{tcolorbox}
    Graph attention networks and graph autoencoder for NLP.\\
    Knowledge graphs or any relational mapping which can be useful for understanding how different tokens
    are related to each other.\\
    Molecules and protein structure applications, because they are structured like a graph, and modeling
    as well as predicting their properties can be useful for drug discovery.\\
    Sociology (like migration between cities) and social network analysis,
    which can be useful for understanding the human world.
\end{tcolorbox}

\pagebreak
\item What questions do you still have about this topic? What resources were helpful in answering these questions?

\begin{tcolorbox}
    How would the implementation and training mechanism of GNNs work? (compared to DNNs) Since GNNs are
    constructed with nodes and edges, traditional techniques like backpropagation could be difficult.
    I also wonder if there are already any industry-level research/application of GNNs that I can look into.
    I want to explroe more about the applications of GNNs in other fields of ML like CV, RL, and NLP because,
    in my perception, DNNs, CNNs, GNNs are like data structures in computer science, and they can all be
    applied to solve a lot of problems.\\
    I want to begin by reading and working through the distill articles and some introductory papers on GNNs,
    I saw that Stanford has a course on GNNs and I will probably take that as well.
\end{tcolorbox}

\item \textbf{WITH YOUR TEAM} Meet (Discord or in person) and summarize the topic that you explored. What was the most interesting? Of all the topics your group explored, do you think you might want to pursue one in particular? Did these topics give you any ideas of your own? Did it reveal any other areas of interest that you have not yet explored?

\begin{tcolorbox}
    Graph neural network. Multi-agent RL. Big data recommendation system.
\end{tcolorbox}

\end{enumerate}
\end{document}