\documentclass[letterpaper,11pt]{article}

\usepackage[shortlabels]{enumitem}
\usepackage[margin=1in]{geometry}
\usepackage[most]{tcolorbox}

\begin{document}
\title{{\bf Module 0: What calls you?} }
\author{Name: Frank Li}

\date{}
\maketitle

\section{Intro}
The purpose of this worksheet is to help spark your imagination and curiousity about the fields of neuroscience, machine learning, general intelligence and creating really cool shit. To get started, I will provide an absolutely $\bf{non-exhaustive}$ list of possible topics, and a few tasks for you to complete.

\subsection{Example Topics}
\begin{itemize}
  \item CS-Focused
  \begin{itemize}
    \item Multi-agent reinforcement learning
        Communication and language between agents, socialization
    \item Machine learning on Intel (or other) neuromorphic chips (Loihi)
    \item Hebbian learning and spiking neural networks
    \item Neural cellular automata
    \item Measuring integrated information across neural networks
  \end{itemize}

  \item Neuro-Focused
  \begin{itemize}
    \item Numenta spiking dendritic networks
    \item Field overview of brain-based intelligent systems
    \item Explore cortical columns and the “Thousand Brains” theory
    \item Explaining how brain structure is related to function in different areas
        Relate this to ML, how can we digitalize these systems?
    \item Biologically inspired forms of memory in AI and reinforcement learning
  \end{itemize}

  \item More General Topics (in case those above seem too specific)
  \begin{itemize}
    \item Neural Network Architecture Optimization
    \item Explainable AI
    \item Computer Vision and correlation with V1 Cortex in the Brain
    \item Stabilizing Deep RL, Pain in RL (MaxPain)
    \item Large Language Models
    \item Multimodal models
    \item Ensemble architectures
    \item Abstract Reasoning Corpus
  \end{itemize}
\end{itemize}
Here are some ways that neuroscience can be used to create generally intelligent machine learning:

\begin{itemize}
  \item Inspiration from the brain: The structure and function of the human brain can inspire the design of machine learning algorithms. For example, deep learning neural networks are inspired by the structure of the brain's neural networks.
  \item Neural network optimization: Machine learning algorithms can be optimized using neural network models that simulate the brain's neural activity. This approach can improve the accuracy and efficiency of machine learning algorithms.
  \item Brain-computer interfaces: Researchers can develop brain-computer interfaces that allow machines to communicate with the brain directly. This technology can enable machines to learn from the brain's neural activity, leading to more intelligent and responsive machine learning algorithms.
  \item Cognitive modeling: Machine learning algorithms can be designed to model cognitive processes, such as attention, memory, and decision-making. This approach can help researchers understand how the brain processes information and develop more intelligent algorithms.
\end{itemize}

\section{Short Answer}
\begin{enumerate}[a)]
\item List at least 5 \textbf{TOPICS OR IDEAS} you have in mind (does not have to be from the list above). If you are having trouble coming up with some, explore google! These are very broad topics. No need to be specific at all yet. An example of this would be: 

FIELD: Reinforcement Learning

\textbf{(15 min)}

\begin{tcolorbox}
  Graph neural networks\\
  Neuromorphic computing\\
  Neural cellular automata\\
  Multi-agent reinforcement learning\\
  Commonsense reasoning
\end{tcolorbox}

\item Create a list of 5-10 \textbf{SUB-FIELDS} from the broad fields listed above. This will require some googling. An example of this would be: 

FIELD: RL

SUB-FIELD: Deep RL in Autonomous robotics and how to ensure safety of these agents

\textbf{(45 min)}

\begin{tcolorbox}
  Graph attention networks\\
  Graph generation\\
  Spiking neural networks\\
  Neuromorphic unsupervised learning\\
  NCA-based algorithms\\
  Reinforcement learning using NCA\\
  Heteregeneous/adversarial multi-agent reinforcement learning\\
  Knowledge graphs
\end{tcolorbox}

For the sub-fields you wrote down, add at least a few sentence description of the field for the five you find most interesting. \textbf{(15 min)}

\textbf{Graph attention networks} - because I am interested in graph neural networks (since it is a novel concept),
and I think the mix of attention and graph neural networks will be very interesting to explore.
I believe GNNs are critical in tackling limitations of traditional neural networks; plus, the graph
structure is more similar to the brain's network structure.

\textbf{Spiking neural networks} - I think it is a very novel concept since traditional neural networks have continuous
data like vectors of numbers but real synapses are discrete and binary. Therefore, implementing a neural network with
discrete inputs similar to real neurons would be a more accurate emulation of the brain.

\textbf{Neuromorphic unsupervised learning} - I feel like this is the closest to emulating the learning process
of a human brain. Loihi also seems like a very intriguing technology and because I like system programming, I am
interested in learning about the hardware and software design of Loihi.

\textbf{NCA-based algorithms} - the article about NCA on distill was very thrilling and I have been wondering if
we could use NCA to optimize different tasks. For example, we could use it as a compression algorithm where a small
number of cells can grow to represent the original data. Modelling cells or physics with NCA could also be useful.

\textbf{Heteregeneous/adversarial multi-agent reinforcement learning} - I am interested in RL with multiple agents
with different goals. For example, in a game of soccer, there are two teams with different goals. I think it would be
cool to simulate interaction different agents either in the form of collaboration or competition. It would be interesting
if we could simulate the process of natural selection.

\item \textbf{WITH YOUR TEAM} please make a list of 10 sub-fields that you all find interesting. This will require some communication! We recommend either meeting up in person or getting on a Discord call. Write the group list down in the box below and \textbf{BOLD} the ones you would like to dive into next week!

\begin{tcolorbox}
Digitalizing the brain\\
Reward function for humans\\
\textbf{Graph attention networks}\\
\textbf{NCA-based algorithms}\\
\textbf{Heteregeneous/adversarial multi-agent reinforcement learning}\\
Multimodality - with videos\\
Transformers for robotics\\
Neurovolume rendering
\end{tcolorbox}

Note that overlaps are okay. Also, be sure to fill this out thoroughly. If you start off weak then you may run into problems in the future!
    
\end{enumerate}
\end{document}